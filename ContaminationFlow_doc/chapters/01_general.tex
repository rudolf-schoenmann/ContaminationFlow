% !TeX root = ../main.tex
% Add the above to each chapter to make compiling the PDF easier in some editors.
\chapter{General  Structure}\label{chapter:general}

%Here comes structure Molflow Linux Molflow Windows, Communication.

\begin{center}
\begin{tikzpicture}[node distance=2cm, font=\scriptsize]
%\draw [help lines] (0,0) grid (4,4);
	\node (Windows)[textbox, rectangle split, rectangle split parts=2] at (0,4)
	{
		\textbf{ContaminationFlow Windows}
		\nodepart{second}
			\begin{itemize}[noitemsep,topsep=0pt,leftmargin=*] \item Graphical User Interface (GUI) \item Create or import Geometry \item Set parameters (temperature, outgassing, etc.) \item Set initial values for coverage \item View profiles \end{itemize}
	};

	\node (Loadbuffer)[textboxsmall, rectangle split, rectangle split parts=2] at (5.5,4)
	{
		\textbf{Loadbuffer}
		\nodepart{second} \begin{itemize}[noitemsep,topsep=0pt,leftmargin=*] \item Geometry \item Parameters  \end{itemize}
	};

	\path let\p1=(Windows.south) in node (Hitbuffer)[textboxsmall, anchor=south, rectangle split, rectangle split parts=2] at (5.5,0.7)
	{
		\textbf{Hitbuffer}
		\nodepart{second} \begin{itemize}[noitemsep,topsep=0pt,leftmargin=*] \item Counters  \end{itemize}
	};

	\node (coveringfile)[textboxsmall, rectangle split, rectangle split parts=2] at (5.5,0)
	{
		\textbf{Covering File}
		\nodepart{second} \begin{itemize}[noitemsep,topsep=0pt,leftmargin=*] \item Set covering or coverage per facet \item Replaces hitbuffer if given \end{itemize}
	};

	\node (inputfile)[textboxsmall, rectangle split, rectangle split parts=2] at (11.5,0)
	{
		\textbf{Input File}
		\nodepart{second} \begin{itemize}[noitemsep,topsep=0pt,leftmargin=*] \item Set simulation paramters \item hitbuffer, loadbuffer, simulation time etc. \end{itemize}
	};

	\node (facetvalues)[textboxsmall, rectangle split, rectangle split parts=2] at (-.5,-0.5)
	{
		\textbf{Simu.\ Parameters}
		\nodepart{second} \begin{itemize}[noitemsep,topsep=0pt,leftmargin=*] \item Export simulation Parameters \item Copied from GUI or saved to text file\end{itemize}
		\raggedright Possible parameters:
		\begin{itemize}[noitemsep,topsep=0pt,leftmargin=*]\item Facet groups \end{itemize}
	};


	\node (Linux)[textbox, rectangle split, rectangle split parts=2] at (11,4)
	{
		\textbf{ContaminationFlow\\Linux}
		\nodepart{second}
			\begin{itemize}[noitemsep,topsep=0pt,leftmargin=*] \item Command line \item Parallel simulation \item Calculates covering, statistical error, particle density, pressure per iteration \item Logarithmic time steps\end{itemize}	
	};

	\draw let \p1=(Windows.east), \p2=(Loadbuffer.second west) in [->, line width=0.04cm] (\x1,\y2) --  (\p2) node[midway,above]{Export};

	\draw let \p1=(Linux.west), \p2=(Loadbuffer.second east) in [<-, line width=0.04cm] (\x1,\y2) --  (\p2)node[midway,above]{Import};

	\draw let \p1=(Windows.east), \p2=(Hitbuffer.second west) in [->, line width=0.04cm] (\x1,\y2+0.1cm) --  ([yshift=0.1cm]\p2)node[midway,above]{Export};
	\draw let \p1=(Linux.west), \p2=(Hitbuffer.second east) in [<-, line width=0.04cm] (\x1,\y2+0.1cm) --  ([yshift=0.1cm]\p2)node[midway,above]{Import};

	\draw let \p1=(Windows.east), \p2=(Hitbuffer.second west) in [<-, line width=0.04cm] (\x1,\y2-0.1cm) --  ([yshift=-0.1cm]\p2)node[midway,below]{Import};

	\draw let \p1=(coveringfile.east), \p2=(Linux.south),\p3=(Linux.west) in [->, line width=0.04cm] (\p1)--(\x3,\y1)node[midway,above]{Import}--(\x2-1.8cm,\y1)--([xshift=-1.8cm]\p2);
	\draw let \p1=(coveringfile.west), \p2=(Windows.south),\p3=(Windows.east) in [<-, line width=0.04cm] (\p1)--(\x3,\y1)node[midway,above]{Export}--(\x2+1.8cm,\y1)--([xshift=1.8cm]\p2);
	
	\draw let \p1=(inputfile.north),\p2=(Linux.south) in  [->, line width=0.04cm] (\p1)--(\x1,\y2)node[midway,right]{Import};

	\draw let \p1=(facetvalues.north),\p2=(Windows.south) in  [<-, line width=0.04cm] (\p1)--(\x1,\y2)node[midway,right]{Export};
		\draw let \p1=(facetvalues.south),\p2=(inputfile.south) in  [->, line width=0.04cm] (\p1)--([yshift=-0.5cm]\p1)--(\x2,\y1-0.5cm)--(\p2)node[midway,right]{Copy};
	
\end{tikzpicture}
\end{center}
\vspace{1cm}
General structure for ContaminationFlow simulation
\begin{itemize}[noitemsep,topsep=2pt]
\item Code adapted from Molflow
\item The Windows executable of ContaminationFlow is used to create Geometry, define the initial value problem and export it via two files (Loadbuffer \& Hitbuffer). It cannot simulate the contamination transfer.
\item The Linux version of ContaminationFlow is used for simulation.
\item Loadbuffer contains information of geometry.
\item Hitbuffer contains information such as hit counters, profiles, etc.
\item Import and export of buffer files are used for communication between ContaminationFlow on Windows and ContaminationFlow on Linux. The import function of the loadbuffer does not work properly yet.
\item Optional: export/import of covering text file that replaces covering in hitbuffer
%\item Import input file to define simulation parameters
%\item Some input file parameters can be exported/copied from ContaminationFlow Windows
%\item Export of SimulationHistory for ContaminationFlow Linux
\end{itemize}