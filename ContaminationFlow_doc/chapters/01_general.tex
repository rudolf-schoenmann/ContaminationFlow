% !TeX root = ../main.tex
% Add the above to each chapter to make compiling the PDF easier in some editors.
\chapter{General  Structure}\label{chapter:general}

%Here comes structure Molflow Linux Molflow Windows, Communication.

\begin{center}
\begin{tikzpicture}[node distance=2cm]
%\draw [help lines] (0,0) grid (4,4);
	\node (Windows)[textbox, rectangle split, rectangle split parts=2] at (0,4)
	{
		\textbf{ContaminationFlow Windows}
		\nodepart{second}
			\begin{itemize}[noitemsep,topsep=0pt,leftmargin=*] \item Main process \item GUI \item Create Geometry \item Set parameters \item View profiles \end{itemize}
	};

	\node (Loadbuffer)[textboxsmall, rectangle split, rectangle split parts=2] at (5.5,4)
	{
		\textbf{Loadbuffer}
		\nodepart{second} \begin{itemize}[noitemsep,topsep=0pt,leftmargin=*] \item Geometry \end{itemize}
	};

	\path let\p1=(Windows.south) in node (Hitbuffer)[textboxsmall, anchor=south, rectangle split, rectangle split parts=2] at (5.5,\y1)
	{
		\textbf{Hitbuffer}
		\nodepart{second} \begin{itemize}[noitemsep,topsep=0pt,leftmargin=*] \item Counters \item etc. \end{itemize}
	};

	\node (Linux)[textbox, rectangle split, rectangle split parts=2] at (11,4)
	{
		\textbf{ContaminationFlow Linux}
		\nodepart{second}
			\begin{itemize}[noitemsep,topsep=0pt,leftmargin=*] \item Worker processes \item Command line \item Parallel simulation \item Update of hitbuffer \end{itemize}	
	};

	%\draw[->, line width=0.04cm] (2.15,3.2) -- (3.85,3.2);
	%\node [above] at (3,3.2) {export};
	\draw let \p1=(Windows.east), \p2=(Loadbuffer.second west) in [->, line width=0.04cm] (\x1,\y2) --  (\p2);
	\path let \p1=(Windows.east), \p2=(Loadbuffer.second west) in node[above] at (0.5*\x1+0.5*\x2,\y2) {Export};

	%\draw[->, line width=0.04cm] (7.15,3.2) -- (8.85,3.2);
	%\node [above] at (8,3.2) {import};
	\draw let \p1=(Linux.west), \p2=(Loadbuffer.second east) in [<-, line width=0.04cm] (\x1,\y2) --  (\p2);
	\path let \p1=(Linux.west), \p2=(Loadbuffer.second east) in node[above] at (0.5*\x1+0.5*\x2,\y2) {Import};

	%\draw[->, line width=0.04cm] (2.15,1.7) -- (3.85,1.7);
	%\node [above] at (3,1.7) {export};
	%\draw[->, line width=0.04cm] (7.15,1.7) -- (8.85,1.7);
	%\node [above] at (8,1.7) {import};
	\draw let \p1=(Windows.east), \p2=(Hitbuffer.second west) in [->, line width=0.04cm] (\x1,\y2+0.1cm) --  ([yshift=0.1cm]\p2);
	\path let \p1=(Windows.east), \p2=(Hitbuffer.second west) in node[above] at (0.5*\x1+0.5*\x2,\y2+0.1cm) {Export};
	\draw let \p1=(Linux.west), \p2=(Hitbuffer.second east) in [<-, line width=0.04cm] (\x1,\y2+0.1cm) --  ([yshift=0.1cm]\p2);
	\path let \p1=(Linux.west), \p2=(Hitbuffer.second east) in node[above] at (0.5*\x1+0.5*\x2,\y2+0.1cm) {Import};

	%\draw[<-, line width=0.04cm] (2.15,1.4) -- (3.85,1.4);
	%\node [below] at (3,1.4) {import};
	%\draw[<-, line width=0.04cm] (7.15,1.4) -- (8.85,1.4);
	%\node [below] at (8,1.4) {export};
	\draw let \p1=(Windows.east), \p2=(Hitbuffer.second west) in [<-, line width=0.04cm] (\x1,\y2-0.1cm) --  ([yshift=-0.1cm]\p2);
	\path let \p1=(Windows.east), \p2=(Hitbuffer.second west) in node[below] at (0.5*\x1+0.5*\x2,\y2-0.1cm) {Import};
	\draw let \p1=(Linux.west), \p2=(Hitbuffer.second east) in [->, line width=0.04cm] (\x1,\y2-0.1cm) --  ([yshift=-0.1cm]\p2);
	\path let \p1=(Linux.west), \p2=(Hitbuffer.second east) in node[below] at (0.5*\x1+0.5*\x2,\y2-0.1cm) {Export};

\end{tikzpicture}
\end{center}
\vspace{1cm}
General structure for ContaminationFlow simulation
\begin{itemize}[noitemsep,topsep=2pt]
\item Code adapted from Molflow
\item ContaminationFlow Windows used to create Geometry 
\item ContaminationFlow Linux used for simulation and calculation of counters, profiles, etc.
\item Loadbuffer contains information of geometry
\item Hitbuffer contains information such as hit counters, profiles, etc.
\item Import and export of buffer files for communication between Molflow Windows and Molflow Linux
\item Import and export of covering history for both linux and windows
\end{itemize}