% !TeX root = ../main.tex
% Add the above to each chapter to make compiling the PDF easier in some editors.

\chapter{ContaminationFlow Windows}\label{chapter:Windows}

\begin{itemize}[noitemsep,topsep=0pt]
\item Create Geometry and set parameters such as pumping speed or sticking
\item Evaluate profiles such as pressure profile
\item Simulation also possible for testing, but mostly done on Linux
\end{itemize}
%\smallskip

\section{Graphical User Interface}
Add screenshot of GUI

\subsubsection{New GUI elements}
\begin{itemize}[noitemsep,topsep=0pt]
\item "Particles out" renamed to Contamination level
\begin{itemize}[noitemsep,topsep=0pt]
\item Text field for covering
\item Text field for coverage
\end{itemize}
\item New facet properties
\begin{itemize}[noitemsep,topsep=0pt]
\item Effective surface factor
\item Facet depth and facet volume
\item Diffusion coefficient
\item Concentration and gas mass
\end{itemize}
%\item Text field for new sticking coefficient
\item Window for CoveringHistory (reworked to SimulationHistory in ContaminationFlow Linux)
\item PressureEvolution window expanded
	\begin{itemize}[noitemsep,topsep=0pt]
	\item Added list that contains information of graph
	\item Option to show only selected facets or all
	\item List exportable
	\end{itemize}
\end{itemize}

%\section{Application}
\section{Communication}
\subsubsection{Import and export of buffer files via GUI}
\begin{itemize}[noitemsep,topsep=0pt]
\item New Databuff struct \code{typedef unsigned char BYTE;\\typedef struct {\\
	\hphantom{\quad}signed int size;\\
	\hphantom{\quad}BYTE *buff;\\
}Databuff;}
\item New functions \codew{importBuff($\cdot$)} and \codew{exportBuff($\cdot$)} for import and export of buffer files/Databuff struct
\item New options in file menu: \codew{Export buffer} and \codew{Import buffer}
\end{itemize}

\section{New Quantities}
\subsubsection{New counter \codew{covering}}
\begin{itemize}[noitemsep,topsep=0pt]
\item Covering computed in \codew{SimulationMC.cpp} file in \codew{updatecovering($\cdot$)}
\item Added covering counter to hitbuffer
\item Added covering to GUI, can be defined through textfield
\end{itemize}

\subsubsection{New facet property \codew{effetiveSurfaceFactor}}
\begin{itemize}[noitemsep,topsep=0pt]
\item Defines increase of facet area due to texture
\end{itemize}


\subsubsection{New facet property \codew{facetDepth}}
\begin{itemize}[noitemsep,topsep=0pt]
\item Defines depth of facet
\end{itemize}


\subsubsection{New facet property \codew{diffusionCoefficient}}
\begin{itemize}[noitemsep,topsep=0pt]
\item Defines diffusion coefficient
\end{itemize}


\subsubsection{New facet property \codew{concentration}}
\begin{itemize}[noitemsep,topsep=0pt]
\item Defines concentration = mass of particles in volume
\end{itemize}

\subsubsection{Removal of irrelevant quantities}
\begin{itemize}[noitemsep,topsep=0pt]
\item Sticking factor and pumping speed removed from GUI
\item \codew{calcSticking()} and \codew{calcFlow()} in \codew{Molflow.cpp} file not used anymore
\item Flow not needed for iterative Algorithm
\end{itemize}

\section{Iterative algorithm}
\subsubsection{New class to store covering for all facets at any time} 
\begin{itemize}[noitemsep,topsep=0pt]
\item In \codew{HistoryWin.cpp} and \codew{HistoryWin.h} file
\item \codew{std::vector<std::pair<double,std::vector<double>\,>\,> pointintime\_list} to store points in time and respective covering for all facets
\item New GUI option to add and remove entries for \codew{pointintime\_list}
\item New GUI option to export or import a complete list
\end{itemize}

%\section{Adaptation of Existing Code}
%\subsubsection{Thats changed}
%\begin{itemize}[noitemsep,topsep=0pt]
%\item use this for 
%\item list
%\end{itemize}
