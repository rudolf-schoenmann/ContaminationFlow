% !TeX root = ../main.tex
% Add the above to each chapter to make compiling the PDF easier in some editors.

\chapter{Conclusion and Outlook}\label{chapter:conclusion}

This report presents the simulation and control of tendon-driven robots with Gazebo and ROS. The MyoMuscle plug-in provided by the Roboy project is introduced. A guide on how to set up the simulation as well as on how to download and install the required tools and code is provided. Additionally, a manual on how to use the MyoMuscle plug-in is provided.

Furthermore, it is shown how new models can be created and processed, such that the MyoMuscle plug-in can be easily added. Using the presented approach, it should in theory be possible to create and simulate different kinds of tendon-driven robots.

However, it is shown that the simulation still has to be improved. Additionally, problems with the creation of new robot models, e.g.\ models with closed geometry, currently occur. Thus, the provided tools have to be refined first before being able to simulate and control all kinds of tendon-driven robots.

Furthermore, as of now, forces can only be send directly to the robot using the MyoMuscle plug-in. Thus, only motor control is currently possible. In real life scenarios, the required forces are not known but rather a desired position, e.g.\ of an arm. This requires the solution of the inverse kinematics problem, which is as of now very hard to solve for tendon-driven robots. Therefore, tools such as CASPR or CASPROS have to be researched in the future in order to enable position control and thus advance the simulation and control (humanoid) tendon-driven robots.

Due to the difficulties arising with tendon-driven robots, the NRP will play an important part in the future. With research concentrating purely on the intersection of computational neuroscience and robotics and the tools provided by the NRP, research and development of simulated, controllable tendon-driven humanoid robots can be facilitated and expedited.

In the future, with an advanced simulation, tendon-driven robots can in turn aid to understand the human body and brain with respective experiments. Once an elaborated control scheme for the simulation exists, it can be transferred to hardware robots to facilitate study of the human body. In the future, it can be possible to develop elaborate brain models or robotic prostheses.