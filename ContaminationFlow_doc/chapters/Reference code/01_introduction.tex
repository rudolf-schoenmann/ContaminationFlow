% !TeX root = ../main.tex
% Add the above to each chapter to make compiling the PDF easier in some editors.
\chapter{Introduction}\label{chapter:introduction}

Humanoid robots are being developed to study the human body, e.g.\ in regard to neuroscience or to develop robotic prostheses, as well as to assist the human, e.g.\ care robots. An  important and difficult research task regarding humanoid robots is how to control them, i.e.\ how to compute the motor commands for the desired movements. 

The Roboy project is an open source project which aims at advancing humanoid robots until they become as good as the human body\cite{Roboy}. For this purpose, several biology-inspired tendon-driven robots, that do not have motors in the joints but use ropes to mimic tendons and muscles, are being developed.%One example is the humanoid tendon-driven robot Roboy.

As hardware setups can be very time consuming, expensive and fragile, it is desirable to execute study and development initially in simulations rather than a real robot.%The simulator that is used in Gazebo, as several models of the Roboy project already exist for it.

To facilitate understanding of the human body and brain, it is adjuvant to provide human-like robots, both real and simulated, as well as means to control them. In case of tendon-driven robots, the ropes have to be pushed or pulled with a certain force. For this purpose, the inverse kinematics problem has to be solved, where the desired position, e.g.\ of a robot arm, is converted to the required forces. Solving the inverse kinematics for tendon-driven robots is a difficult task to solve, as there is a lack of available tools. State-of-the-art programs such as ROS MoveIt\cite{moveit} or IKFast\cite{openrave} fail to solve this problem, as they assume motors in the joints.

One potential solution for the inverse kinematics problem is CASPR. This software should in theory be able to convert the desired angles to motor commands for any tendon-driven robot\cite{CASPR}. This could provide the essential missing link on the way to calculating inverse kinematics for the Roboy project. 

To advance robot simulation, tools for research and analysis have to be provided. The Human brain project flagship researches towards understanding of the structure and organization of the human brain\cite{HBPpaper,HBP}. Neurorobotics is a subproject, which aims at connecting brain models in various virtual environments\cite{NRPpaper}. The Neurorobotics platform (NRP), an open source cloud based system for neurorobotic simulations, provides tools to design and simulate robots as well as to perform and analyze experiments. With these tools, it is possible to simulate brains for virtual and real robots and analyze experiments to advance existing brain models. Thus, the NRP can be essential to study the human body or brain. Biology-inspired spiking neural networks were implemented to learn and execute different types of grasps with an anthropomorphic robot hand\cite{NRPhand} and for model-based polynomial function approximation\cite{NRPsnn}. A hybrid simulation and neuromorphic computing paradigm was developed and demonstrated on a neuromorphic snake-like robot\cite{NRPsnake}. Once these methods are optimized, they can be used to develop brain-inspired controllers.

This report first presents the simulation and control for the Roboy project. The MyoMuscle plug-in\cite{BA} is introduced, which is able to execute motor control. New models for the simulation are created to be integrated in the NRP and it is shown how the MyoMuscle plug-in is added to existing models. Finally, the success and failure as well as gained knowledge of this work is summarized and an outlook is presented.